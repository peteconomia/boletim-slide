% !TeX TS-program = xelatex

\documentclass[12pt,aspectratio=169]{beamer}
\usepackage{graphicx}
\usepackage{xcolor}
\usepackage{fontspec}
\usepackage{tikz}
\usepackage{caption}
\usepackage{subfig}

\setsansfont{EB Garamond}
\definecolor{primarycolor}{RGB}{0, 96, 157}

\setbeamertemplate{itemize item}{\color{primarycolor}$\bullet$}
\setbeamertemplate{itemize subitem}{\color{primarycolor}$\blacktriangleright$}

\usepackage{setspace}
\setstretch{1.3}

\captionsetup{
	format=plain,
	justification=raggedright,
	singlelinecheck=false,
	font={normalsize,color=primarycolor},
	labelfont={color=primarycolor},
	labelsep=space,
	skip=0pt
}

\title{Boletim de Conjuntura Econômica do Tocantins 2020}
\subtitle{v 9, nº1}
\author{PET Economia}
\institute{Universidade Federal do Tocantins}
\date{26/03/2021}

\begin{document}
\setbeamercolor{title}{fg=white}
\setbeamercolor{background canvas}{bg=primarycolor}
\setbeamercolor{normal text}{fg=white}
\usebeamercolor[fg]{normal text}
\begin{frame}
    \begin{tikzpicture}[overlay]
        \node[anchor=center] at (3,-3){\includegraphics[width=7cm]{bg.pdf}};
    \end{tikzpicture}
    \titlepage
\end{frame}

\setbeamercolor{background canvas}{bg=white}
\setbeamercolor{normal text}{fg=black}
\setbeamercolor{title}{fg=red}
\usebeamercolor[fg]{normal text}
\setbeamercolor{frametitle}{fg=primarycolor,bg=white}

\begin{frame}  
\frametitle{Introdução}
\begin{itemize}
\item O Boletim é uma publicação semestral
\item Todo o processo de construção está disponível no GitHub
\item Repositório \url{https://github.com/peteconomia/boletim}
\end{itemize}
\end{frame}

\begin{frame}
    \frametitle{Panorama Econômico}
    \begin{itemize}
        \item XXXX
        \item YYYY
    \end{itemize}
\end{frame}


\begin{frame}
    \frametitle{Contas Públicas Estadual}
    \begin{itemize}
        \item XXXX
        \item YYYY
    \end{itemize}
\end{frame}

\begin{frame}
    \frametitle{Contas Públicas Estadual}
    \begin{figure}%
        \centering
        \subfloat[\centering Despesa com pessoal / RCL]{{\includegraphics[width=.45\textwidth]{figs/desp_pessoal_rcl-1.pdf} }}%
        \qquad
        \subfloat[\centering Dívida Consolidada / RCL]{{\includegraphics[width=.45\textwidth]{figs/divida_rcl-1.pdf} }}%
    \end{figure}
\end{frame}

\end{document}